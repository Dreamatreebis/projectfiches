\documentclass[10pt,a4paper]{article}

%----PACKAGES---
%----layout 
%\usepackage{a4wide}
\usepackage[landscape]{geometry}
\usepackage{float}
\usepackage{framed}

%----figuren en pdfs

\usepackage{graphicx}
\usepackage{pdfpages}
\usepackage{subfigure}


\usepackage{gensymb} %voor \degree (om ° teken te gebruiken)
%----kleur


%\usepackage{color, colortbl}
\usepackage{xcolor}
%\usepackage[dvipsnames]{xcolor}
%\usepackage{color}
\usepackage{soul} % voor het commando: 'highlight' \hl{foo}
\usepackage{colortbl}
\definecolor{Gray}{gray}{0.9}
\definecolor{LightCyan}{rgb}{0.88,1,1}
%\newcommand{\fluo}[1]{$$\fcolorbox{red}{white}{\color{yellow}#1} \rightarrow \text{fluo}	$$ 	}
\newcommand{\mathcolorbox}[2]{\colorbox{#1}{$\displaystyle #2$}}

%----wiskunde
%%%% aanpassing van letter d in afgeleide naar...
\newcommand{\dx}{\hspace{2pt}dx}
\newcommand{\dd}[1]{\hspace{2pt}d#1}
%\usepackage{amsrefs}
\usepackage{amsthm}
\usepackage{amssymb}%nodig om mathbb te kunnen gebruiken,
\usepackage{amsmath}%nodig om zelf math operator te kunnen definiëren
% bijvoorbeeld voor verzameling van reële getallen

%---(hyper)links
\usepackage[pdftex,bookmarks,colorlinks]{hyperref}

%----LOCATIE CONTENT---

\graphicspath{ {./content/} }
%\graphicspath{ {C:/Users/dimitri.coppens/Dropbox/jc_figuren/content/} }
%aparte map voor pdf mogelijk? 
%\graphicspath{ {C:/Users/dimitri.coppens/Dropbox/bubeba_figuren/pdf} }

%-----------------------
%\usepackage[utf8]{inputenc}
\usepackage[utf8]{inputenc}
\usepackage[dutch]{babel}




\usepackage[nottoc]{tocbibind}
\usepackage[none]{hyphenat}


\usepackage{multicol}

%----COMMANDO'S: AANPASSINGEN ---
\renewcommand*\rmdefault{ppl} %héél schoon lettertype :-)
\hypersetup{colorlinks=true,urlcolor=blue}
\renewcommand{\labelitemii}{$\triangleright$}%aanpassing van de 'bullets' 
%\renewcommand{\labelitemii}{\makebox[0pt][l]{$\square$}\raisebox{.15ex}{\hspace{0.1em}$\checkmark$}}%aanpassing van de 'bullets' van naar checkboxes

%--- LAYOUT ---

\setlength{\parindent}{0pt}%zorgt ervoor dat er niet wordt ingesprongen in het begin van een alinea
%aanpassing tekstbreedte en breedte van de kolom in de zijkant
%\hoffset=-0in 

% nota's in zijkanten
%\reversemarginpar
\marginparwidth=2in
\usepackage[paper=A4,pagesize]{typearea}
\usepackage{marginnote}
%\usepackage[top=Bcm, bottom=Hcm, outer=Ccm, inner=Acm, heightrounded, marginparwidth=Ecm, marginparsep=Dcm]{geometry}
%\marginnote{typeset text here...}[Fcm]
%\marginparsep=-1cm

%marginparwidth (E) is the width of the margin note,
%marginparsep (D) is the separation between the paragraph and the margin note,
%F is the downwards vertical offset from the first line the margin note was written (negative values of F shift the margin note upwards), and
%the value G = C − (D + E) is the separation between the edge of the margin note and the edge.

%--- van Marc Coppens



\newtheorem*{voorbeeld}{Voorbeeld} \newtheorem*{eigenschap}{Eigenschap}


%----COMMANDO'S: ZELF GESCHREVEN---
\newcommand{\link}[2]{\color{blue}\underline{\href{#1}{#2}} \color{black}}

\newcommand{\opmerking}[1]{\marginpar{$\leftarrow$ \small{\emph{#1}}}}

\newcommand{\taak}[1]{\marginpar{\vspace{0.5em} #1$\rightarrow$}}
\newcommand{\ok}[0]{\marginpar{ OK$\rightarrow$}}
\newcommand{\nvt}[0]{\marginpar{ NVT$\rightarrow$}}		
\newcommand{\vraag}[1]{\marginpar{\vspace{0.5em} \emph{\footnotesize #1}$\leftarrow$}}
\newcommand{\meerprijs}[1]{\textbf{#1}\marginpar{\emph{meerprijs} $\rightarrow$}}
\newcommand{\minprijs}[1]{\textbf{#1}\marginpar{\emph{minprijs} $\rightarrow$}}
\newcommand{\kost}[1]{\textbf{#1} \marginpar{kost bouwheer}}
\renewcommand{\labelitemi}{$\triangleright$}%aanpassing van de 'bullets' van 'itemize'
%\renewcommand{\labelitemii}{$\circ$}%aanpassing van de 'bullets' van 'itemize'
%\labelenumiii
%\labelenumiv

\newcommand{\mx}[1]{\ensuremath{\mathsf{#1}}}%matrix in juiste lettertype

\DeclareMathOperator{\adj}{adj}
\DeclareMathOperator{\rang}{Rg}

\newcommand{\fout}[1]{$$\fcolorbox{red}{white}{\color{red}#1} \rightarrow \text{fout}	$$ 	}
\newcommand{\juist}[1]{$$\fcolorbox{OliveGreen}{white}{\color{OliveGreen}#1}\rightarrow \text{juist}$$ }
\newcommand{\vakje}[1]{\fcolorbox{blue}{white}{\color{blue}#1}}
%----------------------------------------------------------------voor tex file LIMIETEN ----------------------------------------------

 